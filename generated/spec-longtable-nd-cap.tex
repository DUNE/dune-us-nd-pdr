% N.B. This file is generated, any edits may be lost.
\begin{footnotesize}
\begin{longtable}{p{0.12\textwidth}p{0.18\textwidth}p{0.17\textwidth}p{0.25\textwidth}p{0.16\textwidth}}
\caption{Specifications for ND-CAP \fixmehl{ref \texttt{tab:spec:nd-cap}}} \\
  \rowcolor{dunesky}
       Label & Description  & Specification \newline (Goal) & Rationale & Validation \\  \colhline

\newtag{ND-C1}{ spec:ndlar-reco-nd-vs-fd }
  & ND-LAr particle reconstruction performance
  &  In order to effectively translate measurements to the FD, ND-LAr must be able to reconstruct particles from neutrino events with comparable/better performance than the FD 
  &  
  &  Simulation \\ \colhline
\newtag{ND-C1.1.1}{ spec:ndlar-nue-id }
  & ND-LAr electron neutrino identification
  &  ND-LAr must identify and reconstruct  $\nu_e$ events as well as FD 
  &  ND-LAr nue identification and reconstruction is needed to verify the modelling of the $\nu_e$  flux and cross section, as well as the respone of the FD
  &  Simulation \\ \colhline
\newtag{ND-C1.1.2}{ spec:ndlar-numu-id }
  & ND-LAr muon neutrino identification
  &  ND-LAr must identify and reconstruct  $\nu_\mu$ events as well as FD 
  &  ND-LAr numu identification and reconstruction is needed to verify the modelling of the  $\nu_\mu$ flux and cross section, as well as the respone of the FD
  &  Simulation \\ \colhline
\newtag{ND-C1.1.3}{ spec:ndlar-cont-particle-reco }
  & ND-LAr contained particle reconstruction
  &  Contained particles emerging from ND-LAr should be detected as well as FD. 
  &  ND-LAr particle reconstruction is needed to verify and correct modeling to the extent the the FD observes the corresponding particles.
  &  Simulation \\ \colhline
\newtag{ND-C1.1.4}{ spec:ndlar-limit-uncont-had }
  & ND-LAr limit for uncontained hadron shower
  &  Event topologies/kinematics where no geometric configuration would contain the hadron shower must be limited. 
  &  Uncovered phase space is effectively completely unconstrained by ND-LAr measurements and thus will have a large uncertainty in their contribution to FD distributions. 
  &  Simulation \\ \colhline
\newtag{ND-C1.2.1}{ spec:ndlar-stat-uncert-nue }
  & ND-LAr  statistical uncertainty in electron neutrino measurement
  &  ND-LAr must collect sufficient statistics to allow $<2\%$ statistical uncertainty in the $\nu_e$ measurement 
  &  In order to make a precise measurement, high statistics in this low-cross section process is needed
  &  Simulation \\ \colhline
\newtag{ND-C1.2.2}{ spec:ndlar-id-recoil-e }
  & ND-LAr identification of recoil electron
  &  ND-LAr must be able to identify the recoil electron, distinguish it from other particles ($\mu, \gamma, \pi^0$), and also from $\nu_e$ -CC 
  &  In oder to make a precise measurement, an efficient and pure selection is needed 
  &  Simulation \\ \colhline
\newtag{ND-C1.2.3}{ spec:ndlar-e-energy-res }
  & ND-LAr electron energy resolution
  &  Energy resolution is needed to identify the forward $\nu_e$  events and measure its kinematics 
  &  Both the electron energy and angle are needed in separating this process from the background
  &  Simulation \\ \colhline
\newtag{ND-C1.2.4}{ spec:ndlar-e-ang-res }
  & ND-LAr electron angular resolution
  &  A tight cut on forward electrons is needed to identify $\nu_e$  events 
  &  Both the electron energy and angle are needed in separating this process from the background
  &  Simulation \\ \colhline
\newtag{ND-C1.2.5}{ spec:ndlar-vtx-thresh }
  & ND-LAr vertex activity threshold
  &  Identifying vertex activity near the electron vertex is necessary to reject backgrounds 
  &  $\nu_e$-CC and other backgrounds give rise to hadronic activity near the vertex that is not present for $\nu_e$ elastic events
  &  Simulation \\ \colhline
\newtag{ND-C1.3.1}{ spec:ndlar-scint-time-res }
  & ND-LAr scintillation timing resolution
  &  Scintillation timing is required to set the $t_0$ for the charge readout and to separate event pileup 
  &  The charge readout of ND-LAr occurs on the time scale of ms and does not provide an absolute reference ($t_0$). This is too long to isolate activity from the beam and  obscures the (absolute) coordinate along the drift axis.  The optical detection of scintillation photons must provide both $t_0$ and isolation of the activity from O(100) events spread within the beam delivery window of $\sim$\,\SI{10}{\micro\s}. 
  &   \\ \colhline
\newtag{ND-C1.3.2}{ spec:ndlar-intermod-scint-time }
  & Scintillation timing synchronization between ND-LAr modules
  &  Timing between modules must be synchronized in order to seamlessly integrate activity observed in the separate modules. 
  &  Time  synchronization betweem modules must be comparable to the scintillation timing resolution in order to ensure that event reconstruction across modules is not compromised.
  &   \\ \colhline
\newtag{ND-C2.1}{ spec:tms-ndgar-mu-acc }
  & TMS and ND-GAr muon acceptance
  &  TMS/ND-GAr must detect and analyzes muons exiting the ND-LAr without a gap in phase space coverage 
  &  This completes ND-C1.2 and complements ND-C1.3 by providing the corresponding muon acceptance at higher momentum.
  &  Simulation \\ \colhline
\newtag{ND-C2.2}{ spec:tms-ndgar-mu-res }
  & TMS and ND-GAr momentum resolution
  &  TMS/ND-GAr must measure the muon momentum at least as accurately as the FD 
  &  This completes ND-C1.2 by providing the corresponding muon kinematic reconstruction
  &  Simulation \\ \colhline
\newtag{ND-C2.3}{ spec:tms-ndgar-mu-time }
  & TMS and ND-GAr timing resolution
  &  TMS/ND-GAr must determine the timing of muon tracks in order to separate activity from a single track from other activity in the detector  
  &  The current design of the TMS relies on 2D projective light readout to reduce cost and complexity.  The time resolution for the scintillator trackers must be sufficient to resolve projective ambiguities for a majority of muon signals but will depend on the details of the TMS design.
  &   \\ \colhline
\newtag{ND-C2.4}{ spec:sync-components }
  & Time synchronization of ND components 
  &  TMS/ND-GAr must be synchronized with ND-LAr in order to match activity in ND-LAr with the muon track observed in the TMS/ND-GAr. 
  &  The timing synchrnonization should be such that the timing resolution within the separate systems are not compromised in matching activity across the detectors.
  &   \\ \colhline
\newtag{ND-C3.1}{ spec:ndlar-ndgar-lepton-reco }
  & ND-LAr and ND-GAr lepton momentum and sign reconstruction
  &  Precise lepton momentum/sign reconstruction is needed for detailed kinematic studies, beam $\nu_e$ , and wrong sign measurements 
  &  Precise lepton momentum/sign reconstruction is needed for detailed kinematic studies, beam $\nu_e$, and wrong sign measurements
  &  Simulation \\ \colhline
\newtag{ND-C3.2}{ spec:ndlar-low-energy-pr-reco }
  & ND-LAr low-energy proton reconstruction
  &  Low energy proton reconstruction is needed to verify FSI models and LAr response modelling 
  &  Low threshold proton detection is needed to distinguish between FSI models. Identification of protons is necesary to techniques such as transverse kinematic balance or low-$\nu$.
  &  Simulation \\ \colhline
\newtag{ND-C3.3}{ spec:ndlar-low-energy-pi-reco }
  & ND-LAr low-energy pion reconstruction
  &  Low energy pion reconstruction is needed to verify FSI models and LAr response modeling 
  &  Inefficiency or misidentification of the charged pions can result in biases in neutrino energy reconstruction. To understand this in the context of the LArTPC requires a measurement of the pion spectrum/multiplicity with lower thresholds, sign selection, and minimal secondary interactions that significantly impact the reconstruction of pions in LArTPCs.
  &  Simulation \\ \colhline
\newtag{ND-C3.4}{ spec:ndlar-ch-track-p-res }
  & ND-LAr charged track momentum resolution
  &  Precise momentum resolution of charged recoil particles is needed to study the impact of ND-LAr threshold and measure spectra 
  &  Precise tracking reconstruction is needed to calculate kinematic variables for verifiying and correcting neutrino interaction models
  &  Simulation \\ \colhline
\newtag{ND-C3.5}{ spec:ndlar-ch-part-id }
  & ND-LAr charged particle identification
  &  Recoil particles mut be identified to categorize interactions, tag flavor, and verify modelling of interaction model and ND-LAr response 
  &  Charged particle identification is needed to correctly classify the event in toplogical categories so that model predictons can be verified/corrected.
  &  Simulation \\ \colhline
\newtag{ND-C3.6}{ spec:ndlar-pi-reco }
  & ND-LAr pion reconstruction
  &  $pi^0$'s must be identified/reconstructed to reconstruct the event and have a complete view of pion emission from $\nu$-Ar interactions 
  &  Identification of neutral pions through their decay to two photons is necesary to complete the kinematic reconstruction neutrino interactions to use techniques such as transverse kinematic balance or low-$\nu$
  &  Simulation \\ \colhline
\newtag{ND-C3.7}{ spec:ndgar-ecal-time }
  & ND-GAr ECAL timing
  &  Precise timing is required to provide an absolute reference for the charge signal in the HPgTPC of ND-Gar 
  &  Timing in the ECAL is required to provide an absolute time reference for charge activity observed in the HPgTPC in order to determine the position along the drift axis. There should be both a resolution and efficiency requirement. 
  &  Simulation \\ \colhline
\newtag{ND-C4.1}{ spec:prism-offax-range }
  & PRISM off-axis range
  &  The system should move sufficiently far off-axis to obtain fluxes at the lower range of ND-M7 (id 2483). 
  &  The range of off-axis travel is directly related to the range of spectrum variation. The spectrum variation should be sufficient to cover the range of interest as specified in ND-M7
  &   \\ \colhline
\newtag{ND-C4.2}{ spec:prism-unif-perf }
  & PRISM uniform performance
  &  Uniform performance is needed to make comparative measurements across data taken at different locations 
  &  It is essential that the detectors perform uniformly so that the detected interactions are consistently reconstructed with negligible deviation in detector performance. In addition to any potential effects on the detector components from the motion itself, alignment between the detector elements must be maintained.
  &   \\ \colhline
\newtag{ND-C4.3}{ spec:prism-gran-accur }
  & PRISM granularity and accuracy
  &  Uniform performance is needed to make comparative measurements across data taken at different locations 
  &  In supporting ND-C4.2, it is important that the targeted off-axis position can be placed at the center of the detector so that the fiducial volume can likewise be centered. Precise positioning of the detectors relative to the on-axis location is necessary to accruately determine the flux that is incident on the detector. Furthermore, detector acceptance/performance  varies on the length scale of the ND-LAr module width. Positioning accuracy  and granularity should be much better than this length scale. Similar requirements hold for the TMS/ND-GAr placement
  &   \\ \colhline
\newtag{ND-C4.4}{ spec:prism-min-downtime }
  & PRISM minimize downtime
  &  The ramp down, movement, and ramp up cycle must not cause significant down time, including from farthest off-axis to on-axis. 
  &  Assuming that components of the detector must be ramped down and data-taking stopped during movement, the net down time associated with both the stoppage of data-taking and the movement to a new location must be minimized. The system should further more have the ability to return in the same timeframe to on-axis data-taking to allow regular stability checks. In the worst case, this means moving from the furthest off-axis location to the on-axis location in this time frame, accounting for ramp down/up of the detectors.
  &   \\ \colhline
\newtag{ND-C4.5}{ spec:prism-yearly-meas }
  & PRISM regular suite of measurements
  &  Due to potential variations in the beam line, particulary after regular annual mainetnance periods, the ability to perform measurements each year is needed. 
  &  Due to beamline variations that may be expected, the movement system should allow a complete set of off-axis measurements in the course of an annual run. Summer maintenance periods often result in component replacement which may change the neutrino flux, requiring a new cycle of measurements
  &   \\ \colhline
\newtag{ND-C5.1}{ spec:sand-bm-mon }
  & SAND beam monitoring
  &  SAND must collect and identify enough $\nu_\mu$ CC interactions to perform beam monitoring on a weekly basis 
  &  High statistics, particularly for the spectral analysis, are needed to quickly detect enough $\nu_\mu$-CC events. Some capacity to reconstruct the neutrino energy improves the sensitivity, requiring less statistics.
  &   \\ \colhline
\newtag{ND-C5.2}{ spec:sand-numu-cc-res }
  & SAND resolution for muon neutrino CC events
  &  SAND must sufficient muon moment or neutrino energy resolution to detect spectral variations in $\nu_\mu$ CC events from a representative set of variations in a week 
  &  Simulation of representative beam variations give rise to changes that are on the scale of $\sim$\,\SI{1}{GeV}, which sets the scale of the necessary resolution. The ability to reconstruct neutrino energy with this resolution in addition to the muon improves the sensitivity to potential beam variations; this requires the ability to reconstruct the energy of the hadron system.
  &  Simulation \\ \colhline
\newtag{ND-C5.3}{ spec:sand-vtx-res }
  & SAND vertex resolution
  &  SAND must have the ability to vertex neutrino interactions into upper/lower, left/right regions relative to the nominal beam center 
  &  Some variations result in misdirection of the beam, which can be more readily detected by analyzing the position dependence of the spectrum. Sufficient vertex resolution is needed to separate interactions occuring in each quadrant of the detector transverse to the beam.
  &  Simulation \\ \colhline
\newtag{ND-C5.4}{ spec:sand-track-time }
  & SAND track timing
  &  SAND must have timing to identify and separate activity occuring within the neutrino beam delivery window, at different times within a single beam spill, and reject beam-induced external activity. 
  &  SAND must be able to reject non-beam related activities such as cosmic rays . Due to its proximity to the magnet, which will induce a large number of beam-related backgrounds, the ECAL must have sufficient timing resolution to separate incoming tracks from outgoing ones as they are sampled over the \SI{18}{cm} pathlength.
  &   \\ \colhline

\label{tab:specs:nd-cap}
\end{longtable}
\end{footnotesize}
