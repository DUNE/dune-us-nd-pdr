\chapter{Integration and Installation}
\label{ch:int-inst}

\fixme{Note: the R/O link for the CDR chapter on ND-LAr is https://www.overleaf.com/read/djjrpnnxzbfp}
\fixme{Anne is putting in info from google sheet}
%%%%%%%%%%%%%%%%%%%%%%%%%%%%%%%%
\section{Introduction and Near Detector Facility Layout} 
\label{sec:int-inst-intro}

(The CDR focused a lot on detector details. This content will be reduced in the PDR and replaced with more detailed descriptions of final facility layouts/interfaces and the installation sequence.)

%%%%%%%%%%%%%%%%%%%%%%%%%%%%%%%% 
\section{ND-LAr Detector Layout and Interfaces}
\label{sec:int-inst-ndlar-layout}


%%%%%%%%%%%%%%%%%%%%%%%%%%%%%%%% 
\section{TMS Detector Layout and Interfaces}
\label{sec:int-inst-tms-layout}


%%%%%%%%%%%%%%%%%%%%%%%%%%%%%%%% 
\section{SAND Detector Layout and Interfaces}
\label{sec:int-inst-sand-layout}


%%%%%%%%%%%%%%%%%%%%%%%%%%%%%%%% 
\section{Near Detector Facility Requirements}
\label{sec:int-inst-fac-req}



%%%%%%%%%%%%%%%%%%%%%%%%%%%%%%%% 
\section{Overall Installation Sequence and Schedule}
\label{sec:int-inst-seq-sched}

\fixme{We may want to use a standardized schedule format. (Anne)}

%%%%%%%%%%%%%%%%% from Tim?
\subsection{Standard: SAND First}
\label{sec:int-inst-seq-std}

%%%%%%%%%%%%%%%%% 
\subsection{Backup: SAND Later}
\label{sec:int-inst-seq-bkup}



%%%%%%%%%%%%%%%%%%%%%%%%%%%%%%%%
\section{Interfaces}
\label{sec:int-inst-interface}

\fixme{We may use individual interface tables in the above sections? Anne}
Table~\ref{tbl:int-inst-interfaces} contains a summary and brief description of all the interfaces between the integration and installation consortium and other consortia, working groups, and task forces, with references to the current version of the interface documents describing those interfaces.  
Drawings of the mechanical interfaces and diagrams of the electrical interfaces are 
included in the interface documents as appropriate.
It is expected that further refinements of the interface documents will take place prior to the final \dword{prr} for the detector. The interface documents specify the responsibility of different consortia or groups during all phases of the experiment including design and prototyping, integration,  installation, and  commissioning.


\begin{dunetable}
[Integration and installation interface links]
{p{0.25\textwidth}p{0.5\textwidth}l}
{tbl:int-inst-interfaces}
{Integration and installation interface links}
Interfacing System & Description & Linked Reference \\ \toprowrule
\dword{ndlar}      &  (desc)
& \citedocdb{?} \\ \colhline

\dshort{duneprism} &  (desc)
& \citedocdb{?} \\ \colhline

\dshort{tms}  &  (desc)
& \citedocdb{?} \\ \colhline

and so on     &  (desc)
& \citedocdb{?} \\
\end{dunetable}

%%%%%%%%%%%%%%%%%%%%%%%%%%%%%%%%
\section{Risks and Mitigations}
\label{sec:int-inst-risks}

\fixme{ (This section will also cover delayed detector arrival)}
Table~\ref{tab:risks:int-inst} contains a list of all the
risks that \dword{dune} is currently holding in the I\&I risk register.  Each line includes the official \dword{dune} risk register identification number, a description of the risk, the proposed mitigation for the risk, and finally three columns rating the post-mitigation (P)robability that the risk described comes to pass, the degree of (C)ost risk for that line, and the degree of (S)chedule risk.  Risk levels are defined as (L)ow (<10\% probability of occurring, <5\% cost impact, <2 month schedule impact), (M)edium (10 to 25\% probability of occurring, 5\% to 20\% cost impact, 2 to 6 month schedule impact), or (H)igh (>25\% probability of occurring, >20\% cost impact, >6 month schedule impact).  Most of these risks are reduced to a ``Low'' level following mitigation (as shown in the table), although several of them currently hold a higher risk levels (pre-mitigation), due to the early stage of development of the \dword{ndlar} system relative to other systems.  

In the following sections, we present a narrative description of each of the risks and the proposed mitigation.

\fixme{Anne needs to get risk table template put together}
%\input{generated/risks-longtable-ND-LAr.tex}

\begin{dunetable}
[Placeholder for risks table]
{cc}
{tab:risks:int-inst}
{Placeholder for Risks Table - it will be generated from a spreadsheet}
Rows & Counts \\ \toprowrule
Row 1 & First \\ \colhline
Row 2 & Second \\ \colhline
Row 3 & Third \\ % no \colhline on final row
\end{dunetable}

%%%%%%%%%%%%%%%%%%%%%%%%%%%%%%%% 
\section{Schedule}
\label{sec:int-inst-org-sched}

Table \ref{tab:int-inst-sched} lists key milestones in the design, validation, construction, and installation of the near detector.  These milestones include external milestones indicating linkages to the main \dword{dune} schedule (highlighted in color in the table), as well as internal milestones such as design validation and technical reviews.

\fixme{Anne to get list of main DUNE sched items from Eric J before making the real table template}
\begin{longtable}
{p{0.75\textwidth}p{0.25\textwidth}}
\caption{\dshort{ndlar} consortium schedule}\\ \colhline
\rowcolor{dunetablecolor}Milestone & Date   \\ \toprowrule


\rowcolor{dunepeach}Beneficial occupancy of cavern 1 and \dword{cuc}& \cucbenocc      \\ \colhline
Initial batch (80 PD modules) assembled  & March 2023\\ \colhline

\rowcolor{dunepeach}Top of \dword{detmodule} \#1 cryostat accessible& \accesstopfirstcryo      \\ \colhline
Third batch (320 PD modules) arrive at US PD Reception Facility  & January 2024\\ 

\label{tab:int-inst-sched}
\end{longtable}


%%%%%%%%%%%%%%%%%%%%%%%%%%%%%%%% 
\section{Prototyping Plans}
\label{sec:int-inst-proto}


\begin{comment}
\fixme{The following sections are not in the new structure; I'm keeping them at the end here for now, in case you want to include one or more of them later.}
%%%%%%%%%%%%%%%%%%%%%%%%%%%%%%%% Not in Tim's new organization
\section{Safety Concerns}
\label{sec:int-inst-safety}

%%%%%%%%%%%%%%%%%%%%%%%%%%%%%%%% Not in Tim's new organization
\section{Quality Assurance}
\label{sec:int-inst-qa}

%%%%%%%%%%%%%%%%%%%%%%%%%%%%%%%% Not in Tim's new organization
\section{Transport and Handling}
\label{sec:int-inst-transport}


%%%%%%%%%%%%%%% Not in Tim's new organization
\subsection{Participating Institutions}
\label{sec:fdsp-org-inst}

The \dword{ndlar} consortium benefits from the contributions of many institutions and facilities in \fixme{several countries? or the U.S. and ??}.  Table~\ref{tab:int-inst-institutes}
lists the member institutions. 

\begin{longtable}
{ll}
\caption{\dshort{ndlar} consortium institutions}\\ \colhline
\rowcolor{dunetablecolor} Member Institute  &  Country       \\  \toprowrule
univ 1 &  \\ \colhline
univ 2 &  \\ \colhline
univ 3 &  \\ 
\label{tab:int-inst-institutes}
\end{longtable}

\end{comment}







