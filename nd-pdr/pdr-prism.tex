\chapter{DUNE PRISM}
\label{ch:prism}

%%%%%%%%%%%%%%%%%%%%%%%%%%%%%%%%
\section{Overview of \dshort{duneprism}}
\label{sec:prism-ovvw}

%%%%%%%%%%%%%%%
\subsection{Introduction and Scope}
\label{sec:prism-ovvw-intro}



%%%%%%%%%%%%%%% Not in Tim's new organization
\subsection{Principle of Operation}
\label{sec:prism-ovvw-op}

%%%%%%%%%%%%%%%
\subsection{Design Parameters}
\label{sec:prism-ovvw-param}

\dword{duneprism} is designed to meet the physics requirements of the \dword{dune} experiment. 
The important design parameters are given in Table~\ref{tab:table-prism-params}. 


\begin{dunetable}
[Placeholder for parameter table]
{cc}
{tab:table-prism-params}
{Placeholder for Parameter Table - it will be generated from a spreadsheet}
Rows & Counts \\ \toprowrule
Row 1 & First \\ \colhline
Row 2 & Second \\ \colhline
Row 3 & Third \\ % no \colhline on final row
\end{dunetable}

%%%%%%%%%%%%%%% Not in Tim's new organization
\subsection{Performance}
\label{sec:prism-ovvw-perf}



%%%%%%%%%%%%%%%%%%%%%%%%%%%%%%%%
\section{System Design}
\label{sec:prism-des}

% \begin{enumerate}
%     \item Hilman Roller system
%     \begin{enumerate}
%         \item Commercially available hardware
%         \item Provides precisely controlled movement
%         \item Speed & accelerations controlled by software
%         \begin{enumerate}
%             \item Controlled locally
%             \item Controlled remotely from the control room 
%         \end{enumerate}
%     \item Each unit carries 150 tonnes, 6 units are planned for each detector
%     \item Performance specifications
%     \item Hardware specifications
%     \end{enumerate}
%     \item IGUS Energy Chain
%     \begin{enumerate}
%         \item Commercially available hardware
%         \item Provides a flexible connection to hold, carry, and supply liquids, power, and data lines to the moving detector
%         \item One assembly is used for each detector
%         \item Each unit is oversized (20\% min) to allow for potential future needs
%         \item Performance specifications
%         \item Hardware specifications
%     \end{enumerate}
%     \item Location measurement 
%     \begin{enumerate}
%         \item Counting the Hilman motor pulses defines location
%         \item Use a laser distance measurement for location verification
%         \item Performance specifications
%         \item Hardware specifications
%     \end{enumerate}
%     \item Personnel safety
%     \begin{enumerate}
%         \item Use a safety laser light curtain on each moving detector
%         \item Breaking a beam starts the stop motion sequence
%         \item Performance specifications
%         \item Hardware specifications
%     \end{enumerate}
%     \item System controls
%     \begin{enumerate}
%         \item Automated system control system per industrial standards
%         \item Remote control room monitoring of all systems & detector motion controls
%         \item Safety systems monitored in control room
%     \end{enumerate}
% \end{enumerate}


%%%%%%%%%%%%%%%
\subsection{Moving a Loaded Cryostat}
\label{sec:prism-des-move}

\fixme{Highlight this as unique feature}

%%%%%%%%%%%%%%% One subsection for each WBS element:
\subsection{WBS Element 1}
\label{sec:prism-des-wbs1}

%%%%%%%%%%%%%%% Maybe not needed?
\subsection{Infrastructure and Tooling}
\label{sec:prism-des-infr}

%%%%%%%%%%%%%%% Maybe falls under WBS elements?
\subsection{Auxiliary Pieces and Instrumentation}
\label{sec:prism-des-aux}

%%%%%%%%%%%%%%%%%%%%%%%%%%%%%%%%
\section{Interfaces}
\label{sec:prism-interface}

Table~\ref{tbl:larndinterfaces} contains a summary and brief description of all the interfaces between the \dword{duneprism} consortium and other consortia, working groups, and task forces, with references to the current version of the interface documents describing those interfaces.  
Drawings of the mechanical interfaces and diagrams of the electrical interfaces are 
included in the interface documents as appropriate.
It is expected that further refinements of the interface documents will take place prior to the final \dword{prr} for the detector. The interface documents specify the responsibility of different consortia or groups during all phases of the experiment including design and prototyping, integration,  installation, and  commissioning.


\begin{dunetable}
[\dshort{duneprism} interface links]
{p{0.25\textwidth}p{0.5\textwidth}l}
{tbl:larndinterfaces}
{\dshort{duneprism} interface links}
Interfacing System & Description & Linked Reference \\ \toprowrule
Attachment of rails to the cavern floor     &  (desc)
& \citedocdb{?} \\ \colhline

Energy chains, Connection to cavern walls &  (desc)
& \citedocdb{?} \\ \colhline

Energy chains, Detector services to be run through the chains  &  (desc)
& \citedocdb{?} \\ \colhline

Connection from Hilman skates to detector platforms    &  (desc)
& \citedocdb{?} \\
\end{dunetable}



%%%%%%%%%%%%%%%%%%%%%%%%%%%%%%%%
\section{Risks and Mitigations}
\label{sec:prism-risks}

Table~\ref{tab:table-prism-risks} contains a list of all the
risks that \dword{dune} is currently holding in the \dword{duneprism} risk register.  Each line includes the official \dword{dune} risk register identification number, a description of the risk, the proposed mitigation for the risk, and finally three columns rating the post-mitigation (P)robability that the risk described comes to pass, the degree of (C)ost risk for that line, and the degree of (S)chedule risk.  Risk levels are defined as (L)ow (<10\% probability of occurring, <5\% cost impact, <2 month schedule impact), (M)edium (10 to 25\% probability of occurring, 5\% to 20\% cost impact, 2 to 6 month schedule impact), or (H)igh (>25\% probability of occurring, >20\% cost impact, >6 month schedule impact).  Most of these risks are reduced to a ``Low'' level following mitigation (as shown in the table), although several of them currently hold a higher risk levels (pre-mitigation), due to the early stage of development of the \dword{duneprism} system relative to other systems.  

In the following sections, we present a narrative description of each of the risks and the proposed mitigation.

\fixme{Anne needs to get risk table template put together; below from google sheet}

\begin{itemize}
\item Late delivery of \dword{sand} may affect the run plan
\item Excessive vibrations or accelerations may necessitate additional procedures prior to, and after, movement which can increase downtime and may limit movement frequency
\end{itemize}

%\input{generated/risks-longtable-ND-LAr.tex}

\begin{dunetable}
[Placeholder for risks table]
{cc}
{tab:table-prism-risks}
{Placeholder for Risks Table - it will be generated from a spreadsheet}
Rows & Counts \\ \toprowrule
Row 1 & First \\ \colhline
Row 2 & Second \\ \colhline
Row 3 & Third \\ % no \colhline on final row
\end{dunetable}

%%%%%%%%%%%%%%%%%%%%%%%%%%%%%%%%
\section{Schedule}
\label{sec:prism-org-sched}

Table \ref{tab:prism-sched} lists key milestones in the design, validation, construction, and installation of the \dword{duneprism}.  These milestones include external milestones indicating linkages to the main \dword{dune} schedule (highlighted in color in the table), as well as internal milestones such as design validation and technical reviews.

\fixme{Anne to get list of main DUNE sched items from Eric J before making the real table template}
\begin{longtable}
{p{0.75\textwidth}p{0.25\textwidth}}
\caption{\dshort{duneprism} system consortium schedule}\\ \colhline
\rowcolor{dunetablecolor}Milestone & Date   \\ \toprowrule


\rowcolor{dunepeach}Beneficial occupancy of cavern 1 and \dword{cuc}& \cucbenocc      \\ \colhline
Initial batch (80 PD modules) assembled  & March 2023\\ \colhline

\rowcolor{dunepeach}Top of \dword{detmodule} \#1 cryostat accessible& \accesstopfirstcryo      \\ \colhline
Third batch (320 PD modules) arrive at US PD Reception Facility  & January 2024\\ 

\label{tab:prism-sched}
\end{longtable}

%%%%%%%%%%%%%%%%%%%%%%%%%%%%%%%%
\section{Prototyping Plans}
\label{sec:prism-qc}

\fixme{from google sheet (Anne)}
\begin{itemize}
\item Description of goals

\begin{itemize}
\item Develop control system for the Hilman skates
\item \item Develop monitoring systems
\item \item Determine liquid stability during movement
\item \item Identify relative forces induced on solid submerged structures
\item \item Measure vibration and acceleration levels to inform the development of movement preparation and recovery procedures
\item \item Develop starting and stopping sequences, include gradual (normal) running, and emergency stop procedures
\end{itemize}

\item Test location
\item Personnel
\item Schedule
\end{itemize}



\begin{comment}. %%%%% comment here to end
\fixme{The following sections are not in the new structure; I'm keeping them at the end here for now, in case you want to include one or more of them later.}

%%%%%%%%%%%%%%%%%%%%%%%%%%%%%%%% Not in Tim's new organization
\section{Safety Concerns}
\label{sec:prism-safety}

%%%%%%%%%%%%%%%%%%%%%%%%%%%%%%%%
\section{Calibration}
\label{sec:prism-calib}

%%%%%%%%%%%%%%%%%%%%%%%%%%%%%%%%
\section{Quality Assurance}
\label{sec:prism-qa}

%%%%%%%%%%%%%%%%%%%%%%%%%%%%%%%%
\section{Production, Assembly, and Quality Control}
\label{sec:prism-qc}

%%%%%%%%%%%%%%%%%%%%%%%%%%%%%%%%
\section{Transport and Handling}
\label{sec:prism-transport}

%%%%%%%%%%%%%%%%%%%%%%%%%%%%%%%%
\section{Installation, Integration, and Commissioning}
\label{sec:prism-iic}

%%%%%%%%%%%%%%%%%%%%%%%%%%%%%%%%
\section{Organization}
\label{sec:prism-org}

%%%%%%%%%%%%%%%
\subsection{Participating Institutions}
\label{sec:fdsp-org-inst}
%\metainfo{\color{red}\bf  Content: Segreto/Warner}

The \dword{duneprism} consortium benefits from the contributions of many institutions and facilities in \fixme{several countries? or the U.S. and ??}.  Table~\ref{tab:prism-institutes}
lists the member institutions. 

\begin{longtable}
{ll}
\caption{\dshort{duneprism} consortium institutions}\\ \colhline
\rowcolor{dunetablecolor} Member Institute  &  Country       \\  \toprowrule
univ 1 &  \\ \colhline
univ 2 &  \\ \colhline
univ 3 &  \\ 
\label{tab:prism-institutes}
\end{longtable}

\end{comment}









