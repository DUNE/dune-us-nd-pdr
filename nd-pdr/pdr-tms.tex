\chapter{Tempoary Muon Spectrometer}
\label{ch:tms}



%%%%%%%%%%%%%%%%%%%%%%%%%%%%%%%%
\section{Overview of the TMS}
\label{sec:tms-ovvw}

%%%%%%%%%%%%%%%
\subsection{Introduction and Scope}
\label{sec:tms-ovvw-intro}


%%%%%%%%%%%%%%%Not in Tim's new organization
\subsection{Principle of Operation}
\label{sec:tms-ovvw-op}

%%%%%%%%%%%%%%%
\subsection{Design Parameters}
\label{sec:tms-ovvw-param}

The \dword{tms} is designed to meet the physics requirements of the \dword{dune} experiment. 
The important design parameters are given in Table~\ref{tab:table-tms-params}. 


\begin{dunetable}
[Placeholder for parameter table]
{cc}
{tab:table-tms-params}
{Placeholder for Parameter Table - it will be generated from a spreadsheet}
Rows & Counts \\ \toprowrule
Row 1 & First \\ \colhline
Row 2 & Second \\ \colhline
Row 3 & Third \\ % no \colhline on final row
\end{dunetable}

%%%%%%%%%%%%%%%Not in Tim's new organization
\subsection{Performance}
\label{sec:tms-ovvw-perf}



%%%%%%%%%%%%%%%%%%%%%%%%%%%%%%%%
\section{System Design}
\label{sec:tms-des}

%%%%%%%%%%%%%%%
\subsection{Transition Path to ND-GAr}
\label{sec:tms-des-path}

\fixme{Highlight this as unique feature}

%%%%%%%%%%%%%%% One for each WBS element
\subsection{WBS Element 1}
\label{sec:tms-des-wbs1}

%%%%%%%%%%%%%%%  One for each WBS element
\subsection{Infrastructure and Tooling}
\label{sec:tms-des-infr}

%%%%%%%%%%%%%%% Maybe falls under WBS elements?
\subsection{Auxiliary Pieces and Instrumentation}
\label{sec:tms-des-aux}

%%%%%%%%%%%%%%%%%%%%%%%%%%%%%%%%
\section{Interfaces}
\label{sec:tms-interface}

Table~\ref{tbl:larndinterfaces} contains a summary and brief description of all the interfaces between the \dword{tms} consortium and other consortia, working groups, and task forces, with references to the current version of the interface documents describing those interfaces.  
Drawings of the mechanical interfaces and diagrams of the electrical interfaces are 
included in the interface documents as appropriate.
It is expected that further refinements of the interface documents will take place prior to the final \dword{prr} for the detector. The interface documents specify the responsibility of different consortia or groups during all phases of the experiment including design and prototyping, integration,  installation, and  commissioning.


\begin{dunetable}
[\dshort{tms} interface links]
{p{0.25\textwidth}p{0.5\textwidth}l}
{tbl:larndinterfaces}
{\dshort{tms} interface links}
Interfacing System & Description & Linked Reference \\ \toprowrule
Cryostat      &  (desc)
& \citedocdb{?} \\ \colhline

\dshort{duneprism} &  (desc)
& \citedocdb{?} \\ \colhline

\dshort{tms}  &  (desc)
& \citedocdb{?} \\ \colhline

and so on     &  (desc)
& \citedocdb{?} \\
\end{dunetable}



%%%%%%%%%%%%%%%%%%%%%%%%%%%%%%%%
\section{Risks and Mitigations}
\label{sec:tms-risks}

Table~\ref{tab:risks:ND-LAr} contains a list of all the
risks that \dword{dune} is currently holding in the \dword{tms} risk register.  Each line includes the official \dword{dune} risk register identification number, a description of the risk, the proposed mitigation for the risk, and finally three columns rating the post-mitigation (P)robability that the risk described comes to pass, the degree of (C)ost risk for that line, and the degree of (S)chedule risk.  Risk levels are defined as (L)ow (<10\% probability of occurring, <5\% cost impact, <2 month schedule impact), (M)edium (10 to 25\% probability of occurring, 5\% to 20\% cost impact, 2 to 6 month schedule impact), or (H)igh (>25\% probability of occurring, >20\% cost impact, >6 month schedule impact).  Most of these risks are reduced to a ``Low'' level following mitigation (as shown in the table), although several of them currently hold a higher risk levels (pre-mitigation), due to the early stage of development of the \dword{tms} system relative to other systems.  

In the following sections, we present a narrative description of each of the risks and the proposed mitigation.

\fixme{Anne needs to get risk table template put together}
%\input{generated/risks-longtable-ND-LAr.tex}

\begin{dunetable}
[Placeholder for risks table]
{cc}
{tab:table-tms-risks}
{Placeholder for Risks Table - it will be generated from a spreadsheet}
Rows & Counts \\ \toprowrule
Row 1 & First \\ \colhline
Row 2 & Second \\ \colhline
Row 3 & Third \\ % no \colhline on final row
\end{dunetable}

%%%%%%%%%%%%%%%%%%%%%%%%%%%%%%%%
\section{Schedule}
\label{sec:tms-org-sched}

Table \ref{tab:tms-sched} lists key milestones in the design, validation, construction, and installation of the \dword{tms}.  These milestones include external milestones indicating linkages to the main \dword{dune} schedule (highlighted in color in the table), as well as internal milestones such as design validation and technical reviews.

\fixme{Anne to get list of main DUNE sched items from Eric J before making the real table template}
\begin{longtable}
{p{0.75\textwidth}p{0.25\textwidth}}
\caption{\dshort{tms} consortium schedule}\\ \colhline
\rowcolor{dunetablecolor}Milestone & Date   \\ \toprowrule


\rowcolor{dunepeach}Beneficial occupancy of cavern 1 and \dword{cuc}& \cucbenocc      \\ \colhline
Initial batch (80 PD modules) assembled  & March 2023\\ \colhline

\rowcolor{dunepeach}Top of \dword{detmodule} \#1 cryostat accessible& \accesstopfirstcryo      \\ \colhline
Third batch (320 PD modules) arrive at US PD Reception Facility  & January 2024\\ 

\label{tab:tms-sched}
\end{longtable}

%%%%%%%%%%%%%%%%%%%%%%%%%%%%%%%%
\section{Prototyping Plans}
\label{sec:tms-proto}

%%%%%%%%%%%%%%%%%%%%%%%%%%%%%%%%
\section{Construction Plans}
\label{sec:tms-construc}

\fixme{The following sections are not in the new structure; I'm keeping them at the end here for now, in case you want to include one or more of them later.}

%%%%%%%%%%%%%%%%%%%%%%%%%%%%%%%% Not in Tim's new organization
\section{Safety Concerns}
\label{sec:tms-safety}


%%%%%%%%%%%%%%%%%%%%%%%%%%%%%%%% Not in Tim's new organization
\section{Calibration}
\label{sec:tms-calib}

%%%%%%%%%%%%%%%%%%%%%%%%%%%%%%%% Not in Tim's new organization
\section{Quality Assurance}
\label{sec:tms-qa}

%%%%%%%%%%%%%%%%%%%%%%%%%%%%%%%% Not in Tim's new organization
\section{Production, Assembly, and Quality Control}
\label{sec:tms-qc}

%%%%%%%%%%%%%%%%%%%%%%%%%%%%%%%% Not in Tim's new organization
\section{Transport and Handling}
\label{sec:tms-transport}

%%%%%%%%%%%%%%%%%%%%%%%%%%%%%%%% Not in Tim's new organization
\section{Installation, Integration, and Commissioning}
\label{sec:tms-iic}

%%%%%%%%%%%%%%%%%%%%%%%%%%%%%%%% Not in Tim's new organization
\section{Organization}
\label{sec:tms-org}

%%%%%%%%%%%%%%% Not in Tim's new organization
\subsection{Participating Institutions}
\label{sec:fdsp-org-inst}
%\metainfo{\color{red}\bf  Content: Segreto/Warner}

The \dword{tms} consortium benefits from the contributions of many institutions and facilities in \fixme{several countries? or the U.S. and ??}.  Table~\ref{tab:tms-institutes}
lists the member institutions. 

\begin{longtable}
{ll}
\caption{\dshort{tms} consortium institutions}\\ \colhline
\rowcolor{dunetablecolor} Member Institute  &  Country       \\  \toprowrule
univ 1 &  \\ \colhline
univ 2 &  \\ \colhline
univ 3 &  \\ 
\label{tab:tms-institutes}
\end{longtable}












